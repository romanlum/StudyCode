\documentclass[a4paper]{article}

\usepackage[german]{babel}
\usepackage[utf8]{inputenc}
\usepackage{amsmath}
\usepackage{graphicx}
\usepackage[colorinlistoftodos]{todonotes}

\title{SEb - SWO3 - �bung 1}

\author{Roman Lumetsberger}

\date{\today}

\begin{document}

\section{Aufgabe 1 - Dreieck - Tester}
\subsection{L�sungsidee}
Das Programm braucht genau 3 Eingabewerte, die gepr�ft und auf Flie�kommawerte umgewandelt werden m�ssen.
Folgende �berpr�fungen m�ssen dann mit den Werte durchgef�hrt werden:

\subsubsection{G�ltigkeit des Dreiecks}
Dies kann mit \textbf{Hinweis 1} gepr�ft werden, wobei hier alle Kombinationen getestet werden m�ssen. Es macht aber keinen Unterschied ob man a+b <= c oder b+a <= c betrachtet.\newline
Dadurch ergeben sich genau 3 F�lle:
\begin{itemize}
\item a+b <= c
\item a+c <= b
\item b+c <= a
\end{itemize}

\subsubsection{Gleichseitiges Dreieck}
Ein Dreieck ist gleichseitig, wenn alle 3 Seiten gleich sind.
\begin{itemize}
\item a = b = c
\end{itemize}

\subsubsection{Gleichschenkliges Dreieck}
Ein Dreieck ist gleichschenklig, wenn 2 Seiten gleich sind. Hier ergeben sich wieder 3 F�lle:
\begin{itemize}
\item a = b
\item a = c 
\item b = a
\end{itemize}                             

\subsubsection{Rechtwinkliges Dreieck}
Ob ein Dreieck rechtwinklig ist, kann mit dem Satz von Pythagoras �berpr�ft werden.
Auch hier gibt es wieder 3 F�lle:
\begin{itemize}
\item \(a^2\) + \(b^2\) = \(c^2\)
\item \(a^2\) + \(c^2\) = \(b^2\)
\item \(b^2\) + \(c^2\) = \(a^2\)
\end{itemize}  



\section{Aufgabe 2 - Primfaktorenzerlegung}
\subsection{L�sungsidee}
Das Programm braucht genau einen Eingabewert, der gepr�ft und auf int konvertiert werden muss.
Die Zahl muss gr��er als 0 sein. \newline
Der Algorithmus l�uft dann �ber alle Teiler ab 2 durch und pr�ft, ob die Zahl dividiert durch den Teiler 0 Rest ergiebt.
\begin{itemize}
\item Ist dies der Fall, so wird ein Z�hler, der die Anzahl der Teilungen angibt, erh�ht und die Zahl durch den Teiler dividiert. 
\item Ist dies nicht der Fall, dann muss gepr�ft werden, ob der Z�hler gr��er 0 ist und wenn ja, dann muss der Teiler und Z�hler als Exponentenschreibweise ausgegeben werden. Der Z�hler wird anschlie�end auf 0 gesetzt und der Teiler um 1 erh�ht.
\end{itemize}             
Ist der Teiler gr��er als die Zahl selbst, dann kann der Vorgang beendet werden.


\section{Aufgabe 3 - ERRNO \& Co}
\subsection{L�sungsidee}
Die man-page von atoi erkl�rt, dass \textbf{strtol} als Alternative von atoi verwendet werden kann. Diese Methode funktioniert grunds�tzlich wie atoi, hat aber eine Fehlererkennung.
Diese Methode setzt hierzu die Variable \textbf{errno}, die in der Datei \textit{errno.h} definiert ist.
\subsubsection{Beschreibung ERRNO}
Diese Variable ist in der C-Standardbibliothek \textit{errno.h} definiert und kann verwendet werden um Fehler, die bei einem vorherigen Funktionsaufruf aufgetreten sind, auszuwerten. Dabei ist der Wert nur dann aussagekr�ftig, wenn die aufgerufene Funktion signalisiert, das ein Fehler aufgetreten ist.\newline
Bei \textbf{strol} ist dieser R�ckgabewert 0 \newline
Wichtig ist, dass die Variable vor dem Aufruf auf 0 gesetzt wird und direkt nach dem Aufruf ausgewertet wird.
In der C-Standardbibliothek werden verschiedene Fehlercodes definiert, die in der man-page nachgelesen werden k�nnen.\newline
\subsubsection{Beschreibung strtol}
Diese Funktion konvertiert eine Zeichenkette in einen int.\newline
long int strtol(const char *nptr, char **endptr, int base);\newline
Dabei ist zu beachten, dass die richtige Basis angegeben wird. In diesem Beispiel 10, f�r Dezimal.\newline
Diese Funktion setzt bei einem Fehler die Variable errno und liefert dann als R�ckgabewert 0. Weiters liefert die Funktion einen Zeiger auf eine Zeichenkette \(\textit{endptr}\) zur�ck, die auf das erste ung�ltige Zeichen der Eingabe zeigt.
\begin{itemize}
\item Ist dieser Zeiger gleich mit der Eingabe ==> ganze Zeichenkette ung�ltig
\item Ist das Zeichen an der Zeigerposition '\textbackslash 0' ==> ganze Zeichenkette g�ltig
\end{itemize}
\end{document}
